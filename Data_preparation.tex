% Options for packages loaded elsewhere
\PassOptionsToPackage{unicode}{hyperref}
\PassOptionsToPackage{hyphens}{url}
%
\documentclass[
]{article}
\usepackage{amsmath,amssymb}
\usepackage{lmodern}
\usepackage{ifxetex,ifluatex}
\ifnum 0\ifxetex 1\fi\ifluatex 1\fi=0 % if pdftex
  \usepackage[T1]{fontenc}
  \usepackage[utf8]{inputenc}
  \usepackage{textcomp} % provide euro and other symbols
\else % if luatex or xetex
  \usepackage{unicode-math}
  \defaultfontfeatures{Scale=MatchLowercase}
  \defaultfontfeatures[\rmfamily]{Ligatures=TeX,Scale=1}
  \setmainfont[]{Roboto}
  \setmonofont[]{Consolas}
\fi
% Use upquote if available, for straight quotes in verbatim environments
\IfFileExists{upquote.sty}{\usepackage{upquote}}{}
\IfFileExists{microtype.sty}{% use microtype if available
  \usepackage[]{microtype}
  \UseMicrotypeSet[protrusion]{basicmath} % disable protrusion for tt fonts
}{}
\makeatletter
\@ifundefined{KOMAClassName}{% if non-KOMA class
  \IfFileExists{parskip.sty}{%
    \usepackage{parskip}
  }{% else
    \setlength{\parindent}{0pt}
    \setlength{\parskip}{6pt plus 2pt minus 1pt}}
}{% if KOMA class
  \KOMAoptions{parskip=half}}
\makeatother
\usepackage{xcolor}
\IfFileExists{xurl.sty}{\usepackage{xurl}}{} % add URL line breaks if available
\IfFileExists{bookmark.sty}{\usepackage{bookmark}}{\usepackage{hyperref}}
\hypersetup{
  pdftitle={Quantium Virtual Internship - Retail Strategy and Analytics - Task 1},
  pdfauthor={Paulyna Magana},
  hidelinks,
  pdfcreator={LaTeX via pandoc}}
\urlstyle{same} % disable monospaced font for URLs
\usepackage[margin=1in]{geometry}
\usepackage{color}
\usepackage{fancyvrb}
\newcommand{\VerbBar}{|}
\newcommand{\VERB}{\Verb[commandchars=\\\{\}]}
\DefineVerbatimEnvironment{Highlighting}{Verbatim}{commandchars=\\\{\}}
% Add ',fontsize=\small' for more characters per line
\usepackage{framed}
\definecolor{shadecolor}{RGB}{248,248,248}
\newenvironment{Shaded}{\begin{snugshade}}{\end{snugshade}}
\newcommand{\AlertTok}[1]{\textcolor[rgb]{0.94,0.16,0.16}{#1}}
\newcommand{\AnnotationTok}[1]{\textcolor[rgb]{0.56,0.35,0.01}{\textbf{\textit{#1}}}}
\newcommand{\AttributeTok}[1]{\textcolor[rgb]{0.77,0.63,0.00}{#1}}
\newcommand{\BaseNTok}[1]{\textcolor[rgb]{0.00,0.00,0.81}{#1}}
\newcommand{\BuiltInTok}[1]{#1}
\newcommand{\CharTok}[1]{\textcolor[rgb]{0.31,0.60,0.02}{#1}}
\newcommand{\CommentTok}[1]{\textcolor[rgb]{0.56,0.35,0.01}{\textit{#1}}}
\newcommand{\CommentVarTok}[1]{\textcolor[rgb]{0.56,0.35,0.01}{\textbf{\textit{#1}}}}
\newcommand{\ConstantTok}[1]{\textcolor[rgb]{0.00,0.00,0.00}{#1}}
\newcommand{\ControlFlowTok}[1]{\textcolor[rgb]{0.13,0.29,0.53}{\textbf{#1}}}
\newcommand{\DataTypeTok}[1]{\textcolor[rgb]{0.13,0.29,0.53}{#1}}
\newcommand{\DecValTok}[1]{\textcolor[rgb]{0.00,0.00,0.81}{#1}}
\newcommand{\DocumentationTok}[1]{\textcolor[rgb]{0.56,0.35,0.01}{\textbf{\textit{#1}}}}
\newcommand{\ErrorTok}[1]{\textcolor[rgb]{0.64,0.00,0.00}{\textbf{#1}}}
\newcommand{\ExtensionTok}[1]{#1}
\newcommand{\FloatTok}[1]{\textcolor[rgb]{0.00,0.00,0.81}{#1}}
\newcommand{\FunctionTok}[1]{\textcolor[rgb]{0.00,0.00,0.00}{#1}}
\newcommand{\ImportTok}[1]{#1}
\newcommand{\InformationTok}[1]{\textcolor[rgb]{0.56,0.35,0.01}{\textbf{\textit{#1}}}}
\newcommand{\KeywordTok}[1]{\textcolor[rgb]{0.13,0.29,0.53}{\textbf{#1}}}
\newcommand{\NormalTok}[1]{#1}
\newcommand{\OperatorTok}[1]{\textcolor[rgb]{0.81,0.36,0.00}{\textbf{#1}}}
\newcommand{\OtherTok}[1]{\textcolor[rgb]{0.56,0.35,0.01}{#1}}
\newcommand{\PreprocessorTok}[1]{\textcolor[rgb]{0.56,0.35,0.01}{\textit{#1}}}
\newcommand{\RegionMarkerTok}[1]{#1}
\newcommand{\SpecialCharTok}[1]{\textcolor[rgb]{0.00,0.00,0.00}{#1}}
\newcommand{\SpecialStringTok}[1]{\textcolor[rgb]{0.31,0.60,0.02}{#1}}
\newcommand{\StringTok}[1]{\textcolor[rgb]{0.31,0.60,0.02}{#1}}
\newcommand{\VariableTok}[1]{\textcolor[rgb]{0.00,0.00,0.00}{#1}}
\newcommand{\VerbatimStringTok}[1]{\textcolor[rgb]{0.31,0.60,0.02}{#1}}
\newcommand{\WarningTok}[1]{\textcolor[rgb]{0.56,0.35,0.01}{\textbf{\textit{#1}}}}
\usepackage{graphicx}
\makeatletter
\def\maxwidth{\ifdim\Gin@nat@width>\linewidth\linewidth\else\Gin@nat@width\fi}
\def\maxheight{\ifdim\Gin@nat@height>\textheight\textheight\else\Gin@nat@height\fi}
\makeatother
% Scale images if necessary, so that they will not overflow the page
% margins by default, and it is still possible to overwrite the defaults
% using explicit options in \includegraphics[width, height, ...]{}
\setkeys{Gin}{width=\maxwidth,height=\maxheight,keepaspectratio}
% Set default figure placement to htbp
\makeatletter
\def\fps@figure{htbp}
\makeatother
\setlength{\emergencystretch}{3em} % prevent overfull lines
\providecommand{\tightlist}{%
  \setlength{\itemsep}{0pt}\setlength{\parskip}{0pt}}
\setcounter{secnumdepth}{-\maxdimen} % remove section numbering
\usepackage{fvextra} \DefineVerbatimEnvironment{Highlighting}{Verbatim}{breaklines,commandchars=\\\{\}}
\ifluatex
  \usepackage{selnolig}  % disable illegal ligatures
\fi

\title{Quantium Virtual Internship - Retail Strategy and Analytics -
Task 1}
\author{Paulyna Magana}
\date{}

\begin{document}
\maketitle

\hypertarget{solution-template-for-task-1}{%
\section{Solution template for Task
1}\label{solution-template-for-task-1}}

This file is a solution template for the Task 1 of the Quantium Virtual
Internship. It will walk you through the analysis, providing the
scaffolding for your solution with gaps left for you to fill in
yourself.

\hypertarget{load-required-libraries-and-datasets}{%
\subsection{Load required libraries and
datasets}\label{load-required-libraries-and-datasets}}

\begin{Shaded}
\begin{Highlighting}[]
\DocumentationTok{\#\#\#\# Example code to install packages}
\CommentTok{\#install.packages("data.table")}
\DocumentationTok{\#\#\#\# Load required libraries}
\FunctionTok{library}\NormalTok{(data.table)}
\FunctionTok{library}\NormalTok{(ggplot2)}
\FunctionTok{library}\NormalTok{(ggmosaic)}
\FunctionTok{library}\NormalTok{(readr)}
\FunctionTok{library}\NormalTok{(readxl)}
\FunctionTok{library}\NormalTok{(dplyr)}
\end{Highlighting}
\end{Shaded}

\begin{verbatim}
## 
## Attaching package: 'dplyr'
\end{verbatim}

\begin{verbatim}
## The following objects are masked from 'package:data.table':
## 
##     between, first, last
\end{verbatim}

\begin{verbatim}
## The following objects are masked from 'package:stats':
## 
##     filter, lag
\end{verbatim}

\begin{verbatim}
## The following objects are masked from 'package:base':
## 
##     intersect, setdiff, setequal, union
\end{verbatim}

\begin{Shaded}
\begin{Highlighting}[]
\DocumentationTok{\#\#\#\# Point the filePath to where you have downloaded the datasets to and}
\DocumentationTok{\#\#\#\# assign the data files to data.tables}

\NormalTok{transactionData }\OtherTok{\textless{}{-}} \FunctionTok{data.table}\NormalTok{(}\FunctionTok{read\_excel}\NormalTok{(}\StringTok{"QVI\_transaction\_data.xlsx"}\NormalTok{))}
\NormalTok{customerData }\OtherTok{\textless{}{-}} \FunctionTok{fread}\NormalTok{(}\StringTok{"QVI\_purchase\_behaviour.csv"}\NormalTok{)}
\end{Highlighting}
\end{Shaded}

\hypertarget{exploratory-data-analysis}{%
\subsection{Exploratory data analysis}\label{exploratory-data-analysis}}

The first step in any analysis is to first understand the data. Let's
take a look at each of the datasets provided.

\hypertarget{examining-transaction-data}{%
\subsubsection{Examining transaction
data}\label{examining-transaction-data}}

We can use \texttt{str()} to look at the format of each column and see a
sample of the data. As we have read in the dataset as a
\texttt{data.table} object, we can also run \texttt{transactionData} in
the console to see a sample of the data or use
\texttt{head(transactionData)} to look at the first 10 rows.

Let's check if columns we would expect to be numeric are in numeric form
and date columns are in date format.

\begin{Shaded}
\begin{Highlighting}[]
\DocumentationTok{\#\#\#\# Examine transaction data}
\FunctionTok{str}\NormalTok{(transactionData)}
\end{Highlighting}
\end{Shaded}

\begin{verbatim}
## Classes 'data.table' and 'data.frame': 264836 obs. of 8 variables:
## $ DATE : num 43390 43599 43605 43329 43330 ...
## $ STORE_NBR : num 1 1 1 2 2 4 4 4 5 7 ...
## $ LYLTY_CARD_NBR: num 1000 1307 1343 2373 2426 ...
## $ TXN_ID : num 1 348 383 974 1038 ...
## $ PROD_NBR : num 5 66 61 69 108 57 16 24 42 52 ...
## $ PROD_NAME : chr "Natural Chip Compny SeaSalt175g" "CCs Nacho Cheese 175g"
"Smiths Crinkle Cut Chips Chicken 170g" "Smiths Chip Thinly S/Cream&Onion 175g"
...
## $ PROD_QTY : num 2 3 2 5 3 1 1 1 1 2 ...
## $ TOT_SALES : num 6 6.3 2.9 15 13.8 5.1 5.7 3.6 3.9 7.2 ...
## - attr(*, ".internal.selfref")=<externalptr>
\end{verbatim}

The date column is in an integer format. Let's change this to a date
format.

\begin{Shaded}
\begin{Highlighting}[]
\DocumentationTok{\#\#\#\# Convert DATE column to a date format}
\DocumentationTok{\#\#\#\# A quick search online tells us that CSV and Excel integer dates begin on 30Dec 1899}
\NormalTok{transactionData}\SpecialCharTok{$}\NormalTok{DATE }\OtherTok{\textless{}{-}} \FunctionTok{as.Date}\NormalTok{(transactionData}\SpecialCharTok{$}\NormalTok{DATE, }\AttributeTok{origin =} \StringTok{"1899{-}12{-}30"}\NormalTok{)}
\end{Highlighting}
\end{Shaded}

\begin{Shaded}
\begin{Highlighting}[]
\DocumentationTok{\#\#\#\# Examine transaction data}
\FunctionTok{head}\NormalTok{(transactionData)}
\end{Highlighting}
\end{Shaded}

\begin{verbatim}
##          DATE STORE_NBR LYLTY_CARD_NBR TXN_ID PROD_NBR
## 1: 2018-10-17         1           1000      1        5
## 2: 2019-05-14         1           1307    348       66
## 3: 2019-05-20         1           1343    383       61
## 4: 2018-08-17         2           2373    974       69
## 5: 2018-08-18         2           2426   1038      108
## 6: 2019-05-19         4           4074   2982       57
##                                   PROD_NAME PROD_QTY TOT_SALES
## 1:   Natural Chip        Compny SeaSalt175g        2       6.0
## 2:                 CCs Nacho Cheese    175g        3       6.3
## 3:   Smiths Crinkle Cut  Chips Chicken 170g        2       2.9
## 4:   Smiths Chip Thinly  S/Cream&Onion 175g        5      15.0
## 5: Kettle Tortilla ChpsHny&Jlpno Chili 150g        3      13.8
## 6: Old El Paso Salsa   Dip Tomato Mild 300g        1       5.1
\end{verbatim}

We should check that we are looking at the right products by examining
PROD\_NAME.

\begin{Shaded}
\begin{Highlighting}[]
\DocumentationTok{\#\#\#\# Examine PROD\_NAME}
\NormalTok{transactionData[, .N, PROD\_NAME]}
\end{Highlighting}
\end{Shaded}

\begin{verbatim}
##                                     PROD_NAME    N
##   1:   Natural Chip        Compny SeaSalt175g 1468
##   2:                 CCs Nacho Cheese    175g 1498
##   3:   Smiths Crinkle Cut  Chips Chicken 170g 1484
##   4:   Smiths Chip Thinly  S/Cream&Onion 175g 1473
##   5: Kettle Tortilla ChpsHny&Jlpno Chili 150g 3296
##  ---                                              
## 110:    Red Rock Deli Chikn&Garlic Aioli 150g 1434
## 111:      RRD SR Slow Rst     Pork Belly 150g 1526
## 112:                 RRD Pc Sea Salt     165g 1431
## 113:       Smith Crinkle Cut   Bolognese 150g 1451
## 114:                 Doritos Salsa Mild  300g 1472
\end{verbatim}

Basic text analysis by summarising the individual words in the product
name.

\begin{Shaded}
\begin{Highlighting}[]
\DocumentationTok{\#\#\#\# Examine the words in PROD\_NAME to see if there are any incorrect entries}
\DocumentationTok{\#\#\#\# such as products that are not chips}
\NormalTok{productWords }\OtherTok{\textless{}{-}} \FunctionTok{data.table}\NormalTok{(}\FunctionTok{unlist}\NormalTok{(}\FunctionTok{strsplit}\NormalTok{(}\FunctionTok{unique}\NormalTok{(transactionData[, PROD\_NAME]), }\StringTok{" "}\NormalTok{)))}
\FunctionTok{setnames}\NormalTok{(productWords, }\StringTok{\textquotesingle{}words\textquotesingle{}}\NormalTok{)}
\end{Highlighting}
\end{Shaded}

As we are only interested in words that will tell us if the product is
chips or not, let's remove all words with digits and special characters
such as `\&' from our set of product words. We can do this using
\texttt{grepl()}.

\begin{Shaded}
\begin{Highlighting}[]
\DocumentationTok{\#\#\#\# Removing digits}
\NormalTok{productWords }\OtherTok{\textless{}{-}}\NormalTok{ productWords[}\FunctionTok{grepl}\NormalTok{(}\StringTok{"}\SpecialCharTok{\textbackslash{}\textbackslash{}}\StringTok{d"}\NormalTok{, words) }\SpecialCharTok{==} \ConstantTok{FALSE}\NormalTok{, ]}
\DocumentationTok{\#\#\#\# Removing special characters}
\NormalTok{productWords }\OtherTok{\textless{}{-}}\NormalTok{ productWords[}\FunctionTok{grepl}\NormalTok{(}\StringTok{"[:alpha:]"}\NormalTok{, words), ]}
\DocumentationTok{\#\#\#\# Let\textquotesingle{}s look at the most common words by counting the number of times a wordappears and}
\DocumentationTok{\#\#\#\# sorting them by this frequency in order of highest to lowest frequency}
\NormalTok{productWords[, .N, words][}\FunctionTok{order}\NormalTok{(N, }\AttributeTok{decreasing =} \ConstantTok{TRUE}\NormalTok{)]}
\end{Highlighting}
\end{Shaded}

\begin{verbatim}
##             words  N
##   1:        Chips 21
##   2:       Smiths 16
##   3:      Crinkle 14
##   4:       Kettle 13
##   5:       Cheese 12
##  ---                
## 127: Chikn&Garlic  1
## 128:        Aioli  1
## 129:         Slow  1
## 130:        Belly  1
## 131:    Bolognese  1
\end{verbatim}

There are salsa products in the dataset but we are only interested in
the chips category, so let's remove these.

\begin{Shaded}
\begin{Highlighting}[]
\DocumentationTok{\#\#\#\# Remove salsa products}
\NormalTok{transactionData[, SALSA }\SpecialCharTok{:}\ErrorTok{=} \FunctionTok{grepl}\NormalTok{(}\StringTok{"salsa"}\NormalTok{, }\FunctionTok{tolower}\NormalTok{(PROD\_NAME))]}
\NormalTok{transactionData }\OtherTok{\textless{}{-}}\NormalTok{ transactionData[SALSA }\SpecialCharTok{==} \ConstantTok{FALSE}\NormalTok{, ][, SALSA }\SpecialCharTok{:}\ErrorTok{=} \ConstantTok{NULL}\NormalTok{]}
\end{Highlighting}
\end{Shaded}

Next, we can use \texttt{summary()} to check summary statistics such as
mean, min and max values for each feature to see if there are any
obvious outliers in the data and if there are any nulls in any of the
columns (\texttt{NA\textquotesingle{}s\ :\ number\ of\ nulls} will
appear in the output if there are any nulls).

\begin{Shaded}
\begin{Highlighting}[]
\DocumentationTok{\#\#\#\# Summarise the data to check for nulls and possible outliers}
\FunctionTok{summary}\NormalTok{(transactionData)}
\end{Highlighting}
\end{Shaded}

\begin{verbatim}
##       DATE              STORE_NBR     LYLTY_CARD_NBR        TXN_ID       
##  Min.   :2018-07-01   Min.   :  1.0   Min.   :   1000   Min.   :      1  
##  1st Qu.:2018-09-30   1st Qu.: 70.0   1st Qu.:  70015   1st Qu.:  67569  
##  Median :2018-12-30   Median :130.0   Median : 130367   Median : 135183  
##  Mean   :2018-12-30   Mean   :135.1   Mean   : 135531   Mean   : 135131  
##  3rd Qu.:2019-03-31   3rd Qu.:203.0   3rd Qu.: 203084   3rd Qu.: 202654  
##  Max.   :2019-06-30   Max.   :272.0   Max.   :2373711   Max.   :2415841  
##     PROD_NBR       PROD_NAME            PROD_QTY         TOT_SALES      
##  Min.   :  1.00   Length:246742      Min.   :  1.000   Min.   :  1.700  
##  1st Qu.: 26.00   Class :character   1st Qu.:  2.000   1st Qu.:  5.800  
##  Median : 53.00   Mode  :character   Median :  2.000   Median :  7.400  
##  Mean   : 56.35                      Mean   :  1.908   Mean   :  7.321  
##  3rd Qu.: 87.00                      3rd Qu.:  2.000   3rd Qu.:  8.800  
##  Max.   :114.00                      Max.   :200.000   Max.   :650.000
\end{verbatim}

There are no nulls in the columns but product quantity appears to have
an outlier which we should investigate further. Let's investigate
further the case where 200 packets of chips are bought in one
transaction.

\begin{Shaded}
\begin{Highlighting}[]
\DocumentationTok{\#\#\#\# Filter the dataset to find the outlier}
\NormalTok{outlier }\OtherTok{\textless{}{-}}\NormalTok{ transactionData[PROD\_QTY }\SpecialCharTok{==} \DecValTok{200}\NormalTok{,]}
\end{Highlighting}
\end{Shaded}

There are two transactions where 200 packets of chips are bought in one
transaction and both of these transactions were by the same customer.

\begin{Shaded}
\begin{Highlighting}[]
\DocumentationTok{\#\#\#\# Let\textquotesingle{}s see if the customer has had other transactions}
\NormalTok{transactionData[LYLTY\_CARD\_NBR }\SpecialCharTok{==} \DecValTok{226000}\NormalTok{, ]}
\end{Highlighting}
\end{Shaded}

\begin{verbatim}
##          DATE STORE_NBR LYLTY_CARD_NBR TXN_ID PROD_NBR
## 1: 2018-08-19       226         226000 226201        4
## 2: 2019-05-20       226         226000 226210        4
##                           PROD_NAME PROD_QTY TOT_SALES
## 1: Dorito Corn Chp     Supreme 380g      200       650
## 2: Dorito Corn Chp     Supreme 380g      200       650
\end{verbatim}

It looks like this customer has only had the two transactions over the
year and is not an ordinary retail customer. The customer might be
buying chips for commercial purposes instead. We'll remove this loyalty
card number from further analysis.

\begin{Shaded}
\begin{Highlighting}[]
\DocumentationTok{\#\#\#\# Filter out the customer based on the loyalty card number}
\NormalTok{transactionData }\OtherTok{\textless{}{-}}\NormalTok{ transactionData[LYLTY\_CARD\_NBR }\SpecialCharTok{!=} \DecValTok{226000}\NormalTok{, ]}
\DocumentationTok{\#\#\#\# Re‐examine transaction data}
\FunctionTok{summary}\NormalTok{(transactionData)}
\end{Highlighting}
\end{Shaded}

\begin{verbatim}
##       DATE              STORE_NBR     LYLTY_CARD_NBR        TXN_ID       
##  Min.   :2018-07-01   Min.   :  1.0   Min.   :   1000   Min.   :      1  
##  1st Qu.:2018-09-30   1st Qu.: 70.0   1st Qu.:  70015   1st Qu.:  67569  
##  Median :2018-12-30   Median :130.0   Median : 130367   Median : 135182  
##  Mean   :2018-12-30   Mean   :135.1   Mean   : 135530   Mean   : 135130  
##  3rd Qu.:2019-03-31   3rd Qu.:203.0   3rd Qu.: 203083   3rd Qu.: 202652  
##  Max.   :2019-06-30   Max.   :272.0   Max.   :2373711   Max.   :2415841  
##     PROD_NBR       PROD_NAME            PROD_QTY       TOT_SALES     
##  Min.   :  1.00   Length:246740      Min.   :1.000   Min.   : 1.700  
##  1st Qu.: 26.00   Class :character   1st Qu.:2.000   1st Qu.: 5.800  
##  Median : 53.00   Mode  :character   Median :2.000   Median : 7.400  
##  Mean   : 56.35                      Mean   :1.906   Mean   : 7.316  
##  3rd Qu.: 87.00                      3rd Qu.:2.000   3rd Qu.: 8.800  
##  Max.   :114.00                      Max.   :5.000   Max.   :29.500
\end{verbatim}

That's better. Now, let's look at the number of transaction lines over
time to see if there are any obvious data issues such as missing data.

\begin{Shaded}
\begin{Highlighting}[]
\DocumentationTok{\#\#\#\# Count the number of transactions by date}
\NormalTok{transactionData[, .N, by }\OtherTok{=}\NormalTok{ DATE]}
\end{Highlighting}
\end{Shaded}

\begin{verbatim}
##            DATE   N
##   1: 2018-10-17 682
##   2: 2019-05-14 705
##   3: 2019-05-20 707
##   4: 2018-08-17 663
##   5: 2018-08-18 683
##  ---               
## 360: 2018-12-08 622
## 361: 2019-01-30 689
## 362: 2019-02-09 671
## 363: 2018-08-31 658
## 364: 2019-02-12 684
\end{verbatim}

There's only 364 rows, meaning only 364 dates which indicates a missing
date. Let's create a sequence of dates from 1 Jul 2018 to 30 Jun 2019
and use this to create a chart of number of transactions over time to
find the missing date.

\begin{Shaded}
\begin{Highlighting}[]
\DocumentationTok{\#\#\#\# Create a sequence of dates and join this the count of transactions by date}
\CommentTok{\# Over to you {-} create a column of dates that includes every day from 1 Jul 2018 to}
\CommentTok{\#30 Jun 2019, and join it onto the data to fill in the missing day.}
\NormalTok{allDates }\OtherTok{\textless{}{-}} \FunctionTok{data.table}\NormalTok{(}\FunctionTok{seq}\NormalTok{(}\FunctionTok{as.Date}\NormalTok{(}\StringTok{"2018/07/01"}\NormalTok{), }\FunctionTok{as.Date}\NormalTok{(}\StringTok{"2019/06/30"}\NormalTok{), }\AttributeTok{by =}\StringTok{"day"}\NormalTok{))}
\FunctionTok{setnames}\NormalTok{(allDates, }\StringTok{"DATE"}\NormalTok{)}
\NormalTok{transactions\_by\_day }\OtherTok{\textless{}{-}} \FunctionTok{merge}\NormalTok{(allDates, transactionData[, .N, }\AttributeTok{by =}\NormalTok{ DATE], }\AttributeTok{all.x =} \ConstantTok{TRUE}\NormalTok{)}
\DocumentationTok{\#\#\#\# Setting plot themes to format graphs}
\FunctionTok{theme\_set}\NormalTok{(}\FunctionTok{theme\_bw}\NormalTok{())}
\FunctionTok{theme\_update}\NormalTok{(}\AttributeTok{plot.title =} \FunctionTok{element\_text}\NormalTok{(}\AttributeTok{hjust =} \FloatTok{0.5}\NormalTok{))}
\DocumentationTok{\#\#\#\# Plot transactions over time}
\FunctionTok{ggplot}\NormalTok{(transactions\_by\_day, }\FunctionTok{aes}\NormalTok{(}\AttributeTok{x =}\NormalTok{ DATE, }\AttributeTok{y =}\NormalTok{ N)) }\SpecialCharTok{+}
\FunctionTok{geom\_line}\NormalTok{() }\SpecialCharTok{+}
\FunctionTok{labs}\NormalTok{(}\AttributeTok{x =} \StringTok{"Day"}\NormalTok{, }\AttributeTok{y =} \StringTok{"Number of transactions"}\NormalTok{, }\AttributeTok{title =} \StringTok{"Transactions over time"}\NormalTok{) }\SpecialCharTok{+}
\FunctionTok{scale\_x\_date}\NormalTok{(}\AttributeTok{breaks =} \StringTok{"1 month"}\NormalTok{) }\SpecialCharTok{+}
\FunctionTok{theme}\NormalTok{(}\AttributeTok{axis.text.x =} \FunctionTok{element\_text}\NormalTok{(}\AttributeTok{angle =} \DecValTok{90}\NormalTok{, }\AttributeTok{vjust =} \FloatTok{0.5}\NormalTok{))}
\end{Highlighting}
\end{Shaded}

\begin{center}\includegraphics{Data_preparation_files/figure-latex/unnamed-chunk-7-1} \end{center}

We can see that there is an increase in purchases in December and a
break in late December. Let's zoom in on this.

\begin{Shaded}
\begin{Highlighting}[]
\DocumentationTok{\#\#\#\# Filter to December and look at individual days}
\FunctionTok{ggplot}\NormalTok{(transactions\_by\_day[}\FunctionTok{month}\NormalTok{(DATE) }\SpecialCharTok{==} \DecValTok{12}\NormalTok{, ], }\FunctionTok{aes}\NormalTok{(}\AttributeTok{x =}\NormalTok{ DATE, }\AttributeTok{y =}\NormalTok{ N)) }\SpecialCharTok{+}
\FunctionTok{geom\_line}\NormalTok{() }\SpecialCharTok{+}
\FunctionTok{labs}\NormalTok{(}\AttributeTok{x =} \StringTok{"Day"}\NormalTok{, }\AttributeTok{y =} \StringTok{"Number of transactions"}\NormalTok{, }\AttributeTok{title =} \StringTok{"Transactions over}
\StringTok{↪ time"}\NormalTok{) }\SpecialCharTok{+}
\FunctionTok{scale\_x\_date}\NormalTok{(}\AttributeTok{breaks =} \StringTok{"1 day"}\NormalTok{) }\SpecialCharTok{+}
\FunctionTok{theme}\NormalTok{(}\AttributeTok{axis.text.x =} \FunctionTok{element\_text}\NormalTok{(}\AttributeTok{angle =} \DecValTok{90}\NormalTok{, }\AttributeTok{vjust =} \FloatTok{0.5}\NormalTok{))}
\end{Highlighting}
\end{Shaded}

\begin{verbatim}
## Warning in grid.Call(C_textBounds, as.graphicsAnnot(x$label), x$x, x$y, :
## conversion failure on '↪ time' in 'mbcsToSbcs': dot substituted for <e2>
\end{verbatim}

\begin{verbatim}
## Warning in grid.Call(C_textBounds, as.graphicsAnnot(x$label), x$x, x$y, :
## conversion failure on '↪ time' in 'mbcsToSbcs': dot substituted for <86>
\end{verbatim}

\begin{verbatim}
## Warning in grid.Call(C_textBounds, as.graphicsAnnot(x$label), x$x, x$y, :
## conversion failure on '↪ time' in 'mbcsToSbcs': dot substituted for <aa>
\end{verbatim}

\begin{verbatim}
## Warning in grid.Call(C_textBounds, as.graphicsAnnot(x$label), x$x, x$y, :
## conversion failure on '↪ time' in 'mbcsToSbcs': dot substituted for <e2>
\end{verbatim}

\begin{verbatim}
## Warning in grid.Call(C_textBounds, as.graphicsAnnot(x$label), x$x, x$y, :
## conversion failure on '↪ time' in 'mbcsToSbcs': dot substituted for <86>
\end{verbatim}

\begin{verbatim}
## Warning in grid.Call(C_textBounds, as.graphicsAnnot(x$label), x$x, x$y, :
## conversion failure on '↪ time' in 'mbcsToSbcs': dot substituted for <aa>
\end{verbatim}

\begin{verbatim}
## Warning in grid.Call(C_textBounds, as.graphicsAnnot(x$label), x$x, x$y, :
## conversion failure on '↪ time' in 'mbcsToSbcs': dot substituted for <e2>
\end{verbatim}

\begin{verbatim}
## Warning in grid.Call(C_textBounds, as.graphicsAnnot(x$label), x$x, x$y, :
## conversion failure on '↪ time' in 'mbcsToSbcs': dot substituted for <86>
\end{verbatim}

\begin{verbatim}
## Warning in grid.Call(C_textBounds, as.graphicsAnnot(x$label), x$x, x$y, :
## conversion failure on '↪ time' in 'mbcsToSbcs': dot substituted for <aa>
\end{verbatim}

\begin{verbatim}
## Warning in grid.Call(C_textBounds, as.graphicsAnnot(x$label), x$x, x$y, :
## conversion failure on '↪ time' in 'mbcsToSbcs': dot substituted for <e2>
\end{verbatim}

\begin{verbatim}
## Warning in grid.Call(C_textBounds, as.graphicsAnnot(x$label), x$x, x$y, :
## conversion failure on '↪ time' in 'mbcsToSbcs': dot substituted for <86>
\end{verbatim}

\begin{verbatim}
## Warning in grid.Call(C_textBounds, as.graphicsAnnot(x$label), x$x, x$y, :
## conversion failure on '↪ time' in 'mbcsToSbcs': dot substituted for <aa>
\end{verbatim}

\begin{verbatim}
## Warning in grid.Call(C_textBounds, as.graphicsAnnot(x$label), x$x, x$y, :
## conversion failure on '↪ time' in 'mbcsToSbcs': dot substituted for <e2>
\end{verbatim}

\begin{verbatim}
## Warning in grid.Call(C_textBounds, as.graphicsAnnot(x$label), x$x, x$y, :
## conversion failure on '↪ time' in 'mbcsToSbcs': dot substituted for <86>
\end{verbatim}

\begin{verbatim}
## Warning in grid.Call(C_textBounds, as.graphicsAnnot(x$label), x$x, x$y, :
## conversion failure on '↪ time' in 'mbcsToSbcs': dot substituted for <aa>
\end{verbatim}

\begin{verbatim}
## Warning in grid.Call(C_textBounds, as.graphicsAnnot(x$label), x$x, x$y, :
## conversion failure on '↪ time' in 'mbcsToSbcs': dot substituted for <e2>
\end{verbatim}

\begin{verbatim}
## Warning in grid.Call(C_textBounds, as.graphicsAnnot(x$label), x$x, x$y, :
## conversion failure on '↪ time' in 'mbcsToSbcs': dot substituted for <86>
\end{verbatim}

\begin{verbatim}
## Warning in grid.Call(C_textBounds, as.graphicsAnnot(x$label), x$x, x$y, :
## conversion failure on '↪ time' in 'mbcsToSbcs': dot substituted for <aa>
\end{verbatim}

\begin{verbatim}
## Warning in grid.Call(C_textBounds, as.graphicsAnnot(x$label), x$x, x$y, :
## conversion failure on '↪ time' in 'mbcsToSbcs': dot substituted for <e2>
\end{verbatim}

\begin{verbatim}
## Warning in grid.Call(C_textBounds, as.graphicsAnnot(x$label), x$x, x$y, :
## conversion failure on '↪ time' in 'mbcsToSbcs': dot substituted for <86>
\end{verbatim}

\begin{verbatim}
## Warning in grid.Call(C_textBounds, as.graphicsAnnot(x$label), x$x, x$y, :
## conversion failure on '↪ time' in 'mbcsToSbcs': dot substituted for <aa>
\end{verbatim}

\begin{verbatim}
## Warning in grid.Call.graphics(C_text, as.graphicsAnnot(x$label), x$x, x$y, :
## conversion failure on '↪ time' in 'mbcsToSbcs': dot substituted for <e2>
\end{verbatim}

\begin{verbatim}
## Warning in grid.Call.graphics(C_text, as.graphicsAnnot(x$label), x$x, x$y, :
## conversion failure on '↪ time' in 'mbcsToSbcs': dot substituted for <86>
\end{verbatim}

\begin{verbatim}
## Warning in grid.Call.graphics(C_text, as.graphicsAnnot(x$label), x$x, x$y, :
## conversion failure on '↪ time' in 'mbcsToSbcs': dot substituted for <aa>
\end{verbatim}

\begin{center}\includegraphics{Data_preparation_files/figure-latex/unnamed-chunk-8-1} \end{center}

We can see that the increase in sales occurs in the lead-up to Christmas
and that there are zero sales on Christmas day itself. This is due to
shops being closed on Christmas day. Now that we are satisfied that the
data no longer has outliers, we can move on to creating other features
such as brand of chips or pack size from PROD\_NAME. We will start with
pack size.

\begin{Shaded}
\begin{Highlighting}[]
\DocumentationTok{\#\#\#\# Always check your output}
\NormalTok{transactionData[, PACK\_SIZE }\SpecialCharTok{:}\ErrorTok{=} \FunctionTok{parse\_number}\NormalTok{(PROD\_NAME)]}
\DocumentationTok{\#\#\#\# Let\textquotesingle{}s check if the pack sizes look sensible}
\NormalTok{transactionData[, .N, PACK\_SIZE][}\FunctionTok{order}\NormalTok{(PACK\_SIZE)]}
\end{Highlighting}
\end{Shaded}

\begin{verbatim}
##     PACK_SIZE     N
##  1:        70  1507
##  2:        90  3008
##  3:       110 22387
##  4:       125  1454
##  5:       134 25102
##  6:       135  3257
##  7:       150 40203
##  8:       160  2970
##  9:       165 15297
## 10:       170 19983
## 11:       175 66390
## 12:       180  1468
## 13:       190  2995
## 14:       200  4473
## 15:       210  6272
## 16:       220  1564
## 17:       250  3169
## 18:       270  6285
## 19:       330 12540
## 20:       380  6416
\end{verbatim}

The largest size is 380g and the smallest size is 70g - seems sensible!

\begin{Shaded}
\begin{Highlighting}[]
\DocumentationTok{\#\#\#\# Let\textquotesingle{}s check the output of the first few rows to see if we have indeedpicked out p}
\FunctionTok{head}\NormalTok{(transactionData)}
\end{Highlighting}
\end{Shaded}

\begin{verbatim}
##          DATE STORE_NBR LYLTY_CARD_NBR TXN_ID PROD_NBR
## 1: 2018-10-17         1           1000      1        5
## 2: 2019-05-14         1           1307    348       66
## 3: 2019-05-20         1           1343    383       61
## 4: 2018-08-17         2           2373    974       69
## 5: 2018-08-18         2           2426   1038      108
## 6: 2019-05-16         4           4149   3333       16
##                                   PROD_NAME PROD_QTY TOT_SALES PACK_SIZE
## 1:   Natural Chip        Compny SeaSalt175g        2       6.0       175
## 2:                 CCs Nacho Cheese    175g        3       6.3       175
## 3:   Smiths Crinkle Cut  Chips Chicken 170g        2       2.9       170
## 4:   Smiths Chip Thinly  S/Cream&Onion 175g        5      15.0       175
## 5: Kettle Tortilla ChpsHny&Jlpno Chili 150g        3      13.8       150
## 6: Smiths Crinkle Chips Salt & Vinegar 330g        1       5.7       330
\end{verbatim}

\begin{Shaded}
\begin{Highlighting}[]
\DocumentationTok{\#\#\#\# Let\textquotesingle{}s plot a histogram of PACK\_SIZE since we know that it is a categorical}
\CommentTok{\#variable and not a continuous variable even though it is numeric.}
\FunctionTok{hist}\NormalTok{(transactionData[, PACK\_SIZE])}
\end{Highlighting}
\end{Shaded}

\includegraphics{Data_preparation_files/figure-latex/pack_size histrogram-1.pdf}

Pack sizes created look reasonable. Now to create brands, we can use the
first word in PROD\_NAME to work out the brand name\ldots{}

\begin{Shaded}
\begin{Highlighting}[]
\DocumentationTok{\#\#\#\# Brands}
\NormalTok{transactionData[, BRAND\_NAME }\SpecialCharTok{:}\ErrorTok{=} \FunctionTok{toupper}\NormalTok{(}\FunctionTok{substr}\NormalTok{(PROD\_NAME, }\DecValTok{1}\NormalTok{, }\FunctionTok{regexpr}\NormalTok{(}\AttributeTok{pattern =} \StringTok{\textquotesingle{} \textquotesingle{}}\NormalTok{, PROD\_NAME) }\SpecialCharTok{{-}} \DecValTok{1}\NormalTok{))]}
\DocumentationTok{\#\#\#\# Checking brands}
\NormalTok{transactionData[, .N, by }\OtherTok{=}\NormalTok{ BRAND\_NAME][}\FunctionTok{order}\NormalTok{(}\SpecialCharTok{{-}}\NormalTok{N)]}
\end{Highlighting}
\end{Shaded}

\begin{verbatim}
##     BRAND_NAME     N
##  1:     KETTLE 41288
##  2:     SMITHS 27390
##  3:   PRINGLES 25102
##  4:    DORITOS 22041
##  5:      THINS 14075
##  6:        RRD 11894
##  7:  INFUZIONS 11057
##  8:         WW 10320
##  9:       COBS  9693
## 10:   TOSTITOS  9471
## 11:   TWISTIES  9454
## 12:   TYRRELLS  6442
## 13:      GRAIN  6272
## 14:    NATURAL  6050
## 15:   CHEEZELS  4603
## 16:        CCS  4551
## 17:        RED  4427
## 18:     DORITO  3183
## 19:     INFZNS  3144
## 20:      SMITH  2963
## 21:    CHEETOS  2927
## 22:      SNBTS  1576
## 23:     BURGER  1564
## 24: WOOLWORTHS  1516
## 25:    GRNWVES  1468
## 26:   SUNBITES  1432
## 27:        NCC  1419
## 28:     FRENCH  1418
##     BRAND_NAME     N
\end{verbatim}

\begin{Shaded}
\begin{Highlighting}[]
\DocumentationTok{\#\#\#\# Checking brands}
\CommentTok{\# Over to you! Check the results look reasonable.}
\end{Highlighting}
\end{Shaded}

Some of the brand names look like they are of the same brands - such as
RED and RRD, which are both Red Rock Deli chips. Let's combine these
together.

\begin{Shaded}
\begin{Highlighting}[]
\DocumentationTok{\#\#\#\# Clean brand names}
\NormalTok{transactionData[BRAND\_NAME }\SpecialCharTok{==} \StringTok{"RED"}\NormalTok{, BRAND\_NAME }\SpecialCharTok{:}\ErrorTok{=} \StringTok{"RRD"}\NormalTok{]}
\NormalTok{transactionData[BRAND\_NAME }\SpecialCharTok{==} \StringTok{"GRAIN"}\NormalTok{, BRAND\_NAME }\SpecialCharTok{:}\ErrorTok{=} \StringTok{"GrnWves"}\NormalTok{]}
\NormalTok{transactionData[BRAND\_NAME }\SpecialCharTok{==} \StringTok{"INFZNS"}\NormalTok{, BRAND\_NAME }\SpecialCharTok{:}\ErrorTok{=} \StringTok{"Infuzions"}\NormalTok{]}
\NormalTok{transactionData[BRAND\_NAME }\SpecialCharTok{==} \StringTok{"WW"}\NormalTok{, BRAND\_NAME }\SpecialCharTok{:}\ErrorTok{=} \StringTok{"Woolworths"}\NormalTok{]}
\NormalTok{transactionData[BRAND\_NAME }\SpecialCharTok{==} \StringTok{"SNBTS"}\NormalTok{, BRAND\_NAME }\SpecialCharTok{:}\ErrorTok{=} \StringTok{"Sunbites"}\NormalTok{]}

\DocumentationTok{\#\#\#\# table}
\NormalTok{transactionData[, .N, by }\OtherTok{=}\NormalTok{ BRAND\_NAME][}\FunctionTok{order}\NormalTok{(BRAND\_NAME)]}
\end{Highlighting}
\end{Shaded}

\begin{verbatim}
##     BRAND_NAME     N
##  1:     BURGER  1564
##  2:        CCS  4551
##  3:    CHEETOS  2927
##  4:   CHEEZELS  4603
##  5:       COBS  9693
##  6:     DORITO  3183
##  7:    DORITOS 22041
##  8:     FRENCH  1418
##  9:    GRNWVES  1468
## 10:    GrnWves  6272
## 11:  INFUZIONS 11057
## 12:  Infuzions  3144
## 13:     KETTLE 41288
## 14:    NATURAL  6050
## 15:        NCC  1419
## 16:   PRINGLES 25102
## 17:        RRD 16321
## 18:      SMITH  2963
## 19:     SMITHS 27390
## 20:   SUNBITES  1432
## 21:   Sunbites  1576
## 22:      THINS 14075
## 23:   TOSTITOS  9471
## 24:   TWISTIES  9454
## 25:   TYRRELLS  6442
## 26: WOOLWORTHS  1516
## 27: Woolworths 10320
##     BRAND_NAME     N
\end{verbatim}

\begin{Shaded}
\begin{Highlighting}[]
\DocumentationTok{\#\#\#\# Check again}
\NormalTok{brands }\OtherTok{\textless{}{-}} \FunctionTok{data.frame}\NormalTok{(}\FunctionTok{sort}\NormalTok{(}\FunctionTok{table}\NormalTok{(transactionData}\SpecialCharTok{$}\NormalTok{BRAND\_NAME),}\AttributeTok{decreasing =} \ConstantTok{TRUE}\NormalTok{ ))}

\FunctionTok{setnames}\NormalTok{(brands,}\FunctionTok{c}\NormalTok{(}\StringTok{"BRAND"}\NormalTok{,}\StringTok{"freq"}\NormalTok{))}
\FunctionTok{ggplot}\NormalTok{(brands,}\FunctionTok{aes}\NormalTok{(}\AttributeTok{x=}\NormalTok{BRAND,}\AttributeTok{y=}\NormalTok{ freq,}\AttributeTok{fill=}\NormalTok{BRAND)) }\SpecialCharTok{+}
  \FunctionTok{geom\_bar}\NormalTok{(}\AttributeTok{stat=}\StringTok{"identity"}\NormalTok{,}\AttributeTok{width =} \FloatTok{0.5}\NormalTok{) }\SpecialCharTok{+} 
  \FunctionTok{labs}\NormalTok{(}\AttributeTok{x =} \StringTok{"Brands"}\NormalTok{, }\AttributeTok{y =}\StringTok{"Frequency"}\NormalTok{,}\AttributeTok{title=}\StringTok{"Distribution Of Brand Purchases"}\NormalTok{)}\SpecialCharTok{+}
  \FunctionTok{theme}\NormalTok{(}\AttributeTok{axis.text.x =} \FunctionTok{element\_text}\NormalTok{(}\AttributeTok{angle =} \DecValTok{90}\NormalTok{, }\AttributeTok{vjust =} \FloatTok{0.5}\NormalTok{))}
\end{Highlighting}
\end{Shaded}

\includegraphics{Data_preparation_files/figure-latex/Clean brand names-1.pdf}

\hypertarget{examining-customer-data}{%
\subsubsection{Examining customer data}\label{examining-customer-data}}

Now that we are happy with the transaction dataset, let's have a look at
the customer dataset.

\begin{Shaded}
\begin{Highlighting}[]
\DocumentationTok{\#\#\#\# Examining customer data}
\FunctionTok{summary}\NormalTok{(customerData)}
\end{Highlighting}
\end{Shaded}

\begin{verbatim}
##  LYLTY_CARD_NBR     LIFESTAGE         PREMIUM_CUSTOMER  
##  Min.   :   1000   Length:72637       Length:72637      
##  1st Qu.:  66202   Class :character   Class :character  
##  Median : 134040   Mode  :character   Mode  :character  
##  Mean   : 136186                                        
##  3rd Qu.: 203375                                        
##  Max.   :2373711
\end{verbatim}

\begin{Shaded}
\begin{Highlighting}[]
\FunctionTok{sum}\NormalTok{(}\FunctionTok{is.na}\NormalTok{(customerData))}
\end{Highlighting}
\end{Shaded}

\begin{verbatim}
## [1] 0
\end{verbatim}

\begin{Shaded}
\begin{Highlighting}[]
\NormalTok{lifestageCategory }\OtherTok{\textless{}{-}} \FunctionTok{data.frame}\NormalTok{(}\FunctionTok{sort}\NormalTok{(}\FunctionTok{table}\NormalTok{(customerData}\SpecialCharTok{$}\NormalTok{LIFESTAGE),}\AttributeTok{decreasing =} \ConstantTok{TRUE}\NormalTok{ ))}
\FunctionTok{setnames}\NormalTok{(lifestageCategory,}\FunctionTok{c}\NormalTok{(}\StringTok{"lifestage"}\NormalTok{,}\StringTok{"freq"}\NormalTok{))}


\DocumentationTok{\#\# lifestageCategory}
\FunctionTok{ggplot}\NormalTok{(lifestageCategory,}\FunctionTok{aes}\NormalTok{(}\AttributeTok{x=}\NormalTok{lifestage,}\AttributeTok{y=}\NormalTok{ freq,}\AttributeTok{fill=}\NormalTok{lifestage)) }\SpecialCharTok{+}
  \FunctionTok{geom\_bar}\NormalTok{(}\AttributeTok{stat=}\StringTok{"identity"}\NormalTok{,}\AttributeTok{width =} \FloatTok{0.5}\NormalTok{) }\SpecialCharTok{+} 
  \FunctionTok{labs}\NormalTok{(}\AttributeTok{x =} \StringTok{"lifestage"}\NormalTok{, }\AttributeTok{y =}\StringTok{"frequency"}\NormalTok{,}\AttributeTok{title=}\StringTok{"Distribution Of Customers Over Lifestages"}\NormalTok{)}\SpecialCharTok{+}
  \FunctionTok{theme}\NormalTok{(}\AttributeTok{axis.text.x =} \FunctionTok{element\_text}\NormalTok{(}\AttributeTok{angle =} \DecValTok{90}\NormalTok{, }\AttributeTok{vjust =} \FloatTok{0.5}\NormalTok{))}\SpecialCharTok{+}\FunctionTok{scale\_fill\_brewer}\NormalTok{(}\AttributeTok{palette=}\StringTok{"Dark2"}\NormalTok{)}
\end{Highlighting}
\end{Shaded}

\includegraphics{Data_preparation_files/figure-latex/1 Exploratory data analysis-1.pdf}

\begin{Shaded}
\begin{Highlighting}[]
\NormalTok{premiumCustomerType }\OtherTok{\textless{}{-}} \FunctionTok{data.frame}\NormalTok{(}\FunctionTok{sort}\NormalTok{(}\FunctionTok{table}\NormalTok{(customerData}\SpecialCharTok{$}\NormalTok{PREMIUM\_CUSTOMER),}\AttributeTok{decreasing =} \ConstantTok{TRUE}\NormalTok{ ))}
\FunctionTok{setnames}\NormalTok{(premiumCustomerType,}\FunctionTok{c}\NormalTok{(}\StringTok{"premium\_customer\_type"}\NormalTok{,}\StringTok{"freq"}\NormalTok{))}


\DocumentationTok{\#\#premiumCustomerType}
\FunctionTok{ggplot}\NormalTok{(premiumCustomerType,}\FunctionTok{aes}\NormalTok{(}\AttributeTok{x=}\NormalTok{premium\_customer\_type,}\AttributeTok{y=}\NormalTok{ freq,}\AttributeTok{fill=}\NormalTok{premium\_customer\_type)) }\SpecialCharTok{+}
  \FunctionTok{geom\_bar}\NormalTok{(}\AttributeTok{stat=}\StringTok{"identity"}\NormalTok{,}\AttributeTok{width =} \FloatTok{0.5}\NormalTok{) }\SpecialCharTok{+} 
  \FunctionTok{labs}\NormalTok{(}\AttributeTok{x =} \StringTok{"lifestage"}\NormalTok{, }\AttributeTok{y =}\StringTok{"frequency"}\NormalTok{,}\AttributeTok{title=}\StringTok{"Distribution Of Customers Over Premium Types"}\NormalTok{)}\SpecialCharTok{+}
  \FunctionTok{theme}\NormalTok{(}\AttributeTok{axis.text.x =} \FunctionTok{element\_text}\NormalTok{(}\AttributeTok{angle =} \DecValTok{90}\NormalTok{, }\AttributeTok{vjust =} \FloatTok{0.5}\NormalTok{))}\SpecialCharTok{+}\FunctionTok{scale\_fill\_brewer}\NormalTok{(}\AttributeTok{palette=}\StringTok{"Dark2"}\NormalTok{)}
\end{Highlighting}
\end{Shaded}

\includegraphics{Data_preparation_files/figure-latex/1 Exploratory data analysis-2.pdf}

\begin{Shaded}
\begin{Highlighting}[]
\DocumentationTok{\#\#\#\# Merge transaction data to customer data}
\NormalTok{data }\OtherTok{\textless{}{-}} \FunctionTok{merge}\NormalTok{(transactionData, customerData, }\AttributeTok{all.x =} \ConstantTok{TRUE}\NormalTok{)}
\end{Highlighting}
\end{Shaded}

As the number of rows in \texttt{data} is the same as that of
\texttt{transactionData}, we can be sure that no duplicates were
created. This is because we created \texttt{data} by setting
\texttt{all.x\ =\ TRUE} (in other words, a left join) which means take
all the rows in \texttt{transactionData} and find rows with matching
values in shared columns and then joining the details in these rows to
the \texttt{x} or the first mentioned table.

Let's also check if some customers were not matched on by checking for
nulls.

\begin{Shaded}
\begin{Highlighting}[]
\FunctionTok{sum}\NormalTok{(}\FunctionTok{is.na}\NormalTok{(data))}
\end{Highlighting}
\end{Shaded}

\begin{verbatim}
## [1] 0
\end{verbatim}

Great, there are no nulls! So all our customers in the transaction data
has been accounted for in the customer dataset. Note that if you are
continuing with Task 2, you may want to retain this dataset which you
can write out as a csv

\begin{Shaded}
\begin{Highlighting}[]
\FunctionTok{fwrite}\NormalTok{(data, }\StringTok{"QVI\_data.csv"}\NormalTok{)}
\end{Highlighting}
\end{Shaded}

Data exploration is now complete! \#\# Data analysis on customer
segments Now that the data is ready for analysis, we can define some
metrics of interest to the client: - Who spends the most on chips (total
sales), describing customers by lifestage and how premium their general
purchasing behaviour is - How many customers are in each segment - How
many chips are bought per customer by segment - What's the average chip
price by customer segment We could also ask our data team for more
information. Examples are: - The customer's total spend over the period
and total spend for each transaction to understand what proportion of
their grocery spend is on chips - Proportion of customers in each
customer segment overall to compare against the mix of customers who
purchase chips Let's start with calculating total sales by LIFESTAGE and
PREMIUM\_CUSTOMER and plotting the split by these segments to describe
which customer segment contribute most to chip sales.

\begin{Shaded}
\begin{Highlighting}[]
\DocumentationTok{\#\#\#\# Total sales by LIFESTAGE and PREMIUM\_CUSTOMER}
\NormalTok{sales }\OtherTok{\textless{}{-}}\NormalTok{ data[, .(}\AttributeTok{SALES =} \FunctionTok{sum}\NormalTok{(TOT\_SALES)), .(LIFESTAGE, PREMIUM\_CUSTOMER)]}
\DocumentationTok{\#\#\#\# Create plot}
\NormalTok{p }\OtherTok{\textless{}{-}} \FunctionTok{ggplot}\NormalTok{(}\AttributeTok{data =}\NormalTok{ sales) }\SpecialCharTok{+}
\FunctionTok{geom\_mosaic}\NormalTok{(}\FunctionTok{aes}\NormalTok{(}\AttributeTok{weight =}\NormalTok{ SALES, }\AttributeTok{x =} \FunctionTok{product}\NormalTok{(PREMIUM\_CUSTOMER, LIFESTAGE), }\AttributeTok{fill =}\NormalTok{ PREMIUM\_CUSTOMER)) }\SpecialCharTok{+}
\FunctionTok{labs}\NormalTok{(}\AttributeTok{x =} \StringTok{"Lifestage"}\NormalTok{, }\AttributeTok{y =} \StringTok{"Premium customer flag"}\NormalTok{, }\AttributeTok{title =} \StringTok{"Proportion of sales"}\NormalTok{) }\SpecialCharTok{+}
\FunctionTok{theme}\NormalTok{(}\AttributeTok{axis.text.x =} \FunctionTok{element\_text}\NormalTok{(}\AttributeTok{angle =} \DecValTok{90}\NormalTok{, }\AttributeTok{vjust =} \FloatTok{0.5}\NormalTok{))}
\DocumentationTok{\#\#\#\# Plot and label with proportion of sales}
\NormalTok{p }\SpecialCharTok{+} \FunctionTok{geom\_text}\NormalTok{(}\AttributeTok{data =} \FunctionTok{ggplot\_build}\NormalTok{(p)}\SpecialCharTok{$}\NormalTok{data[[}\DecValTok{1}\NormalTok{]], }\FunctionTok{aes}\NormalTok{(}\AttributeTok{x =}\NormalTok{ (xmin }\SpecialCharTok{+}\NormalTok{ xmax)}\SpecialCharTok{/}\DecValTok{2}\NormalTok{ , }\AttributeTok{y =}
\NormalTok{(ymin }\SpecialCharTok{+}\NormalTok{ ymax)}\SpecialCharTok{/}\DecValTok{2}\NormalTok{, }\AttributeTok{label =} \FunctionTok{as.character}\NormalTok{(}\FunctionTok{paste}\NormalTok{(}\FunctionTok{round}\NormalTok{(.wt}\SpecialCharTok{/}\FunctionTok{sum}\NormalTok{(.wt),}\DecValTok{3}\NormalTok{)}\SpecialCharTok{*}\DecValTok{100}\NormalTok{,}
\StringTok{\textquotesingle{}\%\textquotesingle{}}\NormalTok{))))}
\end{Highlighting}
\end{Shaded}

\begin{verbatim}
## Warning: `unite_()` was deprecated in tidyr 1.2.0.
## Please use `unite()` instead.
## This warning is displayed once every 8 hours.
## Call `lifecycle::last_lifecycle_warnings()` to see where this warning was generated.
\end{verbatim}

\begin{center}\includegraphics{Data_preparation_files/figure-latex/unnamed-chunk-10-1} \end{center}

Sales are coming mainly from Budget - older families, Mainstream - young
singles/couples, and Mainstream - retirees Let's see if the higher sales
are due to there being more customers who buy chips.

\begin{Shaded}
\begin{Highlighting}[]
\DocumentationTok{\#\#\#\# Number of customers by LIFESTAGE and PREMIUM\_CUSTOMER}
\NormalTok{customers }\OtherTok{\textless{}{-}}\NormalTok{ data[, .(}\AttributeTok{CUSTOMERS =} \FunctionTok{uniqueN}\NormalTok{(LYLTY\_CARD\_NBR)), .(LIFESTAGE, PREMIUM\_CUSTOMER)][}\FunctionTok{order}\NormalTok{(}\SpecialCharTok{{-}}\NormalTok{CUSTOMERS)]}
\DocumentationTok{\#\#\#\# Create plot}
\NormalTok{p }\OtherTok{\textless{}{-}} \FunctionTok{ggplot}\NormalTok{(}\AttributeTok{data =}\NormalTok{ customers) }\SpecialCharTok{+}
\FunctionTok{geom\_mosaic}\NormalTok{(}\FunctionTok{aes}\NormalTok{(}\AttributeTok{weight =}\NormalTok{ CUSTOMERS, }\AttributeTok{x =} \FunctionTok{product}\NormalTok{(PREMIUM\_CUSTOMER, LIFESTAGE), }\AttributeTok{fill =}\NormalTok{ PREMIUM\_CUSTOMER)) }\SpecialCharTok{+}
\FunctionTok{labs}\NormalTok{(}\AttributeTok{x =} \StringTok{"Lifestage"}\NormalTok{, }\AttributeTok{y =} \StringTok{"Premium customer flag"}\NormalTok{, }\AttributeTok{title =} \StringTok{"Proportion of customers"}\NormalTok{) }\SpecialCharTok{+}
\FunctionTok{theme}\NormalTok{(}\AttributeTok{axis.text.x =} \FunctionTok{element\_text}\NormalTok{(}\AttributeTok{angle =} \DecValTok{90}\NormalTok{, }\AttributeTok{vjust =} \FloatTok{0.5}\NormalTok{))}
\DocumentationTok{\#\#\#\# Plot and label with proportion of customers}
\NormalTok{p }\SpecialCharTok{+} \FunctionTok{geom\_text}\NormalTok{(}\AttributeTok{data =} \FunctionTok{ggplot\_build}\NormalTok{(p)}\SpecialCharTok{$}\NormalTok{data[[}\DecValTok{1}\NormalTok{]], }\FunctionTok{aes}\NormalTok{(}\AttributeTok{x =}\NormalTok{ (xmin }\SpecialCharTok{+}\NormalTok{ xmax)}\SpecialCharTok{/}\DecValTok{2}\NormalTok{ , }\AttributeTok{y =}
\NormalTok{(ymin }\SpecialCharTok{+}\NormalTok{ ymax)}\SpecialCharTok{/}\DecValTok{2}\NormalTok{, }\AttributeTok{label =} \FunctionTok{as.character}\NormalTok{(}\FunctionTok{paste}\NormalTok{(}\FunctionTok{round}\NormalTok{(.wt}\SpecialCharTok{/}\FunctionTok{sum}\NormalTok{(.wt),}\DecValTok{3}\NormalTok{)}\SpecialCharTok{*}\DecValTok{100}\NormalTok{,}
\StringTok{\textquotesingle{}\%\textquotesingle{}}\NormalTok{))))}
\end{Highlighting}
\end{Shaded}

\begin{center}\includegraphics{Data_preparation_files/figure-latex/unnamed-chunk-11-1} \end{center}

There are more Mainstream - young singles/couples and Mainstream -
retirees who buy chips. This contributes to there being more sales to
these customer segments but this is not a major driver for the Budget -
Older families segment. Higher sales may also be driven by more units of
chips being bought per customer. Let's have a look at this next.

\begin{Shaded}
\begin{Highlighting}[]
\DocumentationTok{\#\#\#\# Average number of units per customer by LIFESTAGE and PREMIUM\_CUSTOMER}
\NormalTok{avg\_units }\OtherTok{\textless{}{-}}\NormalTok{ data[, .(}\AttributeTok{AVG =} \FunctionTok{sum}\NormalTok{(PROD\_QTY)}\SpecialCharTok{/}\FunctionTok{uniqueN}\NormalTok{(LYLTY\_CARD\_NBR)),.(LIFESTAGE, PREMIUM\_CUSTOMER)][}\FunctionTok{order}\NormalTok{(}\SpecialCharTok{{-}}\NormalTok{AVG)]}
\DocumentationTok{\#\#\#\# Create plot}
\FunctionTok{ggplot}\NormalTok{(}\AttributeTok{data =}\NormalTok{ avg\_units, }\FunctionTok{aes}\NormalTok{(}\AttributeTok{weight =}\NormalTok{ AVG, }\AttributeTok{x =}\NormalTok{ LIFESTAGE, }\AttributeTok{fill =}\NormalTok{PREMIUM\_CUSTOMER)) }\SpecialCharTok{+}
\FunctionTok{geom\_bar}\NormalTok{(}\AttributeTok{position =} \FunctionTok{position\_dodge}\NormalTok{()) }\SpecialCharTok{+}
\FunctionTok{labs}\NormalTok{(}\AttributeTok{x =} \StringTok{"Lifestage"}\NormalTok{, }\AttributeTok{y =} \StringTok{"Avg units per transaction"}\NormalTok{, }\AttributeTok{title =} \StringTok{"Units percustomer"}\NormalTok{) }\SpecialCharTok{+}
\FunctionTok{theme}\NormalTok{(}\AttributeTok{axis.text.x =} \FunctionTok{element\_text}\NormalTok{(}\AttributeTok{angle =} \DecValTok{90}\NormalTok{, }\AttributeTok{vjust =} \FloatTok{0.5}\NormalTok{))}
\end{Highlighting}
\end{Shaded}

\begin{center}\includegraphics{Data_preparation_files/figure-latex/unnamed-chunk-12-1} \end{center}

Older families and young families in general buy more chips per customer
Let's also investigate the average price per unit chips bought for each
customer segment as this is also a driver of total sales.

\begin{Shaded}
\begin{Highlighting}[]
\DocumentationTok{\#\#\#\# Average price per unit by LIFESTAGE and PREMIUM\_CUSTOMER}
\NormalTok{avg\_price }\OtherTok{\textless{}{-}}\NormalTok{ data[, .(}\AttributeTok{AVG =} \FunctionTok{sum}\NormalTok{(TOT\_SALES)}\SpecialCharTok{/}\FunctionTok{sum}\NormalTok{(PROD\_QTY)), .(LIFESTAGE,PREMIUM\_CUSTOMER)][}\FunctionTok{order}\NormalTok{(}\SpecialCharTok{{-}}\NormalTok{AVG)]}
\DocumentationTok{\#\#\#\# Create plot}
\FunctionTok{ggplot}\NormalTok{(}\AttributeTok{data =}\NormalTok{ avg\_price, }\FunctionTok{aes}\NormalTok{(}\AttributeTok{weight =}\NormalTok{ AVG, }\AttributeTok{x =}\NormalTok{ LIFESTAGE, }\AttributeTok{fill =}\NormalTok{PREMIUM\_CUSTOMER)) }\SpecialCharTok{+}
\FunctionTok{geom\_bar}\NormalTok{(}\AttributeTok{position =} \FunctionTok{position\_dodge}\NormalTok{()) }\SpecialCharTok{+}
\FunctionTok{labs}\NormalTok{(}\AttributeTok{x =} \StringTok{"Lifestage"}\NormalTok{, }\AttributeTok{y =} \StringTok{"Avg price per unit"}\NormalTok{, }\AttributeTok{title =} \StringTok{"Price per unit"}\NormalTok{) }\SpecialCharTok{+}
\FunctionTok{theme}\NormalTok{(}\AttributeTok{axis.text.x =} \FunctionTok{element\_text}\NormalTok{(}\AttributeTok{angle =} \DecValTok{90}\NormalTok{, }\AttributeTok{vjust =} \FloatTok{0.5}\NormalTok{))}
\end{Highlighting}
\end{Shaded}

\begin{center}\includegraphics{Data_preparation_files/figure-latex/unnamed-chunk-13-1} \end{center}

Mainstream midage and young singles and couples are more willing to pay
more per packet of chips compared to their budget and premium
counterparts. This may be due to premium shoppers being more likely to
buy healthy snacks and when they buy chips, this is mainly for
entertainment purposes rather than their own consumption. This is also
supported by there being fewer premium midage and young singles and
couples buying chips compared to their mainstream counterparts. As the
difference in average price per unit isn't large, we can check if this

difference is statistically different.

\begin{Shaded}
\begin{Highlighting}[]
\DocumentationTok{\#\#\#\# Perform an independent t{-}test between mainstream vs premium and budget midageand}
\DocumentationTok{\#\#\#\# young singles and couples}
\NormalTok{pricePerUnit }\OtherTok{\textless{}{-}}\NormalTok{ data[, price }\SpecialCharTok{:}\ErrorTok{=}\NormalTok{ TOT\_SALES}\SpecialCharTok{/}\NormalTok{PROD\_QTY]}
\FunctionTok{t.test}\NormalTok{(data[LIFESTAGE }\SpecialCharTok{\%in\%} \FunctionTok{c}\NormalTok{(}\StringTok{"YOUNG SINGLES/COUPLES"}\NormalTok{, }\StringTok{"MIDAGE SINGLES/COUPLES"}\NormalTok{) }\SpecialCharTok{\&}\NormalTok{ PREMIUM\_CUSTOMER }\SpecialCharTok{==} \StringTok{"Mainstream"}\NormalTok{, price], data[LIFESTAGE }\SpecialCharTok{\%in\%} \FunctionTok{c}\NormalTok{(}\StringTok{"YOUNG SINGLES/COUPLES"}\NormalTok{, }\StringTok{"MIDAGE SINGLES/COUPLES"}\NormalTok{) }\SpecialCharTok{\&}\NormalTok{ PREMIUM\_CUSTOMER }\SpecialCharTok{!=} \StringTok{"Mainstream"}\NormalTok{, price], }\AttributeTok{alternative =} \StringTok{"greater"}\NormalTok{)}
\end{Highlighting}
\end{Shaded}

\begin{verbatim}
##
## Welch Two Sample t-test
##
## data: data[LIFESTAGE %in% c("YOUNG SINGLES/COUPLES", "MIDAGE
SINGLES/COUPLES") & PREMIUM_CUSTOMER == "Mainstream", price] and data[LIFESTAGE
%in% c("YOUNG SINGLES/COUPLES", "MIDAGE SINGLES/COUPLES") & PREMIUM_CUSTOMER !=
"Mainstream", price]
## t = 37.624, df = 54791, p-value < 2.2e-16
## alternative hypothesis: true difference in means is greater than 0
## 95 percent confidence interval:
## 0.3187234 Inf
## sample estimates:
## mean of x mean of y
## 4.039786 3.706491
\end{verbatim}

The t-test results in a p-value of XXXXXXX, i.e.~the unit price for
mainstream, young and mid-age singles and couples {[}ARE / ARE NOT{]}
significantly higher than that of budget or premium, young and midage
singles and couples. \#\# Deep dive into specific customer segments for
insights We have found quite a few interesting insights that we can dive
deeper into. We might want to target customer segments that contribute
the most to sales to retain them or further increase sales. Let's look
at Mainstream - young singles/couples. For instance, let's find out if
they tend to buy a particular brand of chips.

\begin{Shaded}
\begin{Highlighting}[]
\DocumentationTok{\#\#\#\# Deep dive into Mainstream, young singles/couples}
\NormalTok{segment1}\OtherTok{\textless{}{-}}\NormalTok{ data[LIFESTAGE }\SpecialCharTok{==} \StringTok{"YOUNG SINGLES/COUPLES"} \SpecialCharTok{\&}\NormalTok{ PREMIUM\_CUSTOMER }\SpecialCharTok{==} \StringTok{"Mainstream"}\NormalTok{,]}
\NormalTok{other }\OtherTok{\textless{}{-}}\NormalTok{ data[}\SpecialCharTok{!}\NormalTok{(LIFESTAGE }\SpecialCharTok{==} \StringTok{"YOUNG SINGLES/COUPLES"} \SpecialCharTok{\&}\NormalTok{ PREMIUM\_CUSTOMER }\SpecialCharTok{==} \StringTok{"Mainstream"}\NormalTok{),]}
\DocumentationTok{\#\#\#\# Brand affinity compared to the rest of the population}
\NormalTok{quantity\_segment1 }\OtherTok{\textless{}{-}}\NormalTok{ segment1[, }\FunctionTok{sum}\NormalTok{(PROD\_QTY)]}

\NormalTok{quantity\_other }\OtherTok{\textless{}{-}}\NormalTok{ other[, }\FunctionTok{sum}\NormalTok{(PROD\_QTY)]}
\NormalTok{quantity\_segment1\_by\_brand }\OtherTok{\textless{}{-}}\NormalTok{ segment1[, .(}\AttributeTok{targetSegment =} \FunctionTok{sum}\NormalTok{(PROD\_QTY)}\SpecialCharTok{/}\NormalTok{quantity\_segment1), by }\OtherTok{=}\NormalTok{ BRAND\_NAME]}
\NormalTok{quantity\_other\_by\_brand }\OtherTok{\textless{}{-}}\NormalTok{ other[, .(}\AttributeTok{other =} \FunctionTok{sum}\NormalTok{(PROD\_QTY)}\SpecialCharTok{/}\NormalTok{quantity\_other), by }\OtherTok{=}\NormalTok{ BRAND\_NAME]}
\NormalTok{brand\_proportions }\OtherTok{\textless{}{-}} \FunctionTok{merge}\NormalTok{(quantity\_segment1\_by\_brand, quantity\_other\_by\_brand)[, affinityToBrand }\SpecialCharTok{:}\ErrorTok{=}\NormalTok{ targetSegment}\SpecialCharTok{/}\NormalTok{other]}
\NormalTok{brand\_proportions[}\FunctionTok{order}\NormalTok{(}\SpecialCharTok{{-}}\NormalTok{affinityToBrand)]}
\end{Highlighting}
\end{Shaded}

\begin{verbatim}
##     BRAND_NAME targetSegment       other affinityToBrand
##  1:     DORITO   0.015707384 0.012759861       1.2309996
##  2:   TYRRELLS   0.031552795 0.025692464       1.2280953
##  3:   TWISTIES   0.046183575 0.037876520       1.2193194
##  4:    DORITOS   0.107053140 0.088314823       1.2121764
##  5:     KETTLE   0.197984817 0.165553442       1.1958967
##  6:   TOSTITOS   0.045410628 0.037977861       1.1957131
##  7:  Infuzions   0.014934438 0.012573300       1.1877898
##  8:   PRINGLES   0.119420290 0.100634769       1.1866703
##  9:    GrnWves   0.029123533 0.025121265       1.1593180
## 10:       COBS   0.044637681 0.039048861       1.1431238
## 11:  INFUZIONS   0.049744651 0.044491379       1.1180739
## 12:      THINS   0.060372671 0.056986370       1.0594230
## 13:   CHEEZELS   0.017971014 0.018646902       0.9637534
## 14:     SMITHS   0.089772257 0.112215379       0.7999996
## 15:     FRENCH   0.003947550 0.005758060       0.6855694
## 16:    CHEETOS   0.008033126 0.012066591       0.6657329
## 17:        RRD   0.043809524 0.067493678       0.6490908
## 18:    NATURAL   0.015955832 0.024980768       0.6387246
## 19:        NCC   0.003643892 0.005873221       0.6204248
## 20:        CCS   0.011180124 0.018895650       0.5916771
## 21:    GRNWVES   0.003588682 0.006066692       0.5915385
## 22:      SMITH   0.006597654 0.012368313       0.5334320
## 23:   Sunbites   0.003478261 0.006587221       0.5280316
## 24: Woolworths   0.021256039 0.043049561       0.4937574
## 25:   SUNBITES   0.002870945 0.005992989       0.4790507
## 26: WOOLWORTHS   0.002843340 0.006377627       0.4458304
## 27:     BURGER   0.002926156 0.006596434       0.4435967
##     BRAND_NAME targetSegment       other affinityToBrand
\end{verbatim}

We can see that : • Mainstream young singles/couples are 23\% more
likely to purchase Tyrrells chips compared to the rest of the population
• Mainstream young singles/couples are 56\% less likely to buy Burger
Rings compared to the rest of the population Let's also find out if our
target segment tends to purchase larger packs of chips.

Let's also find out if our target segment tends to buy larger packs of
chips.

\begin{Shaded}
\begin{Highlighting}[]
\DocumentationTok{\#\#\#\# Preferred pack size compared to the rest of the population}
\NormalTok{quantity\_segment1\_by\_pack }\OtherTok{\textless{}{-}}\NormalTok{ segment1[, .(}\AttributeTok{targetSegment =} \FunctionTok{sum}\NormalTok{(PROD\_QTY)}\SpecialCharTok{/}\NormalTok{quantity\_segment1), by }\OtherTok{=}\NormalTok{ PACK\_SIZE]}
\NormalTok{quantity\_other\_by\_pack }\OtherTok{\textless{}{-}}\NormalTok{ other[, .(}\AttributeTok{other =} \FunctionTok{sum}\NormalTok{(PROD\_QTY)}\SpecialCharTok{/}\NormalTok{quantity\_other), by }\OtherTok{=}\NormalTok{PACK\_SIZE]}
\NormalTok{pack\_proportions }\OtherTok{\textless{}{-}} \FunctionTok{merge}\NormalTok{(quantity\_segment1\_by\_pack, quantity\_other\_by\_pack)[, affinityToPack }\SpecialCharTok{:}\ErrorTok{=}\NormalTok{ targetSegment}\SpecialCharTok{/}\NormalTok{other]}
\NormalTok{pack\_proportions[}\FunctionTok{order}\NormalTok{(}\SpecialCharTok{{-}}\NormalTok{affinityToPack)]}
\end{Highlighting}
\end{Shaded}

\begin{verbatim}
##     PACK_SIZE targetSegment       other affinityToPack
##  1:       270   0.031828847 0.025095929      1.2682873
##  2:       380   0.032160110 0.025584213      1.2570295
##  3:       330   0.061283644 0.050161917      1.2217166
##  4:       134   0.119420290 0.100634769      1.1866703
##  5:       110   0.106280193 0.089791190      1.1836372
##  6:       210   0.029123533 0.025121265      1.1593180
##  7:       135   0.014768806 0.013075403      1.1295106
##  8:       250   0.014354727 0.012780590      1.1231662
##  9:       170   0.080772947 0.080985964      0.9973697
## 10:       150   0.157598344 0.163420656      0.9643722
## 11:       175   0.254989648 0.270006956      0.9443818
## 12:       165   0.055652174 0.062267662      0.8937572
## 13:       190   0.007481021 0.012442016      0.6012708
## 14:       180   0.003588682 0.006066692      0.5915385
## 15:       160   0.006404417 0.012372920      0.5176157
## 16:        90   0.006349206 0.012580210      0.5046980
## 17:       125   0.003008972 0.006036750      0.4984423
## 18:       200   0.008971705 0.018656115      0.4808989
## 19:        70   0.003036577 0.006322350      0.4802924
## 20:       220   0.002926156 0.006596434      0.4435967
\end{verbatim}

{[}INSIGHTS{]}

\begin{enumerate}
\def\labelenumi{\arabic{enumi}.}
\item
  Sales have mainly been due to Budget - older families, Mainstream -
  young singles/couples, and Mainstream- retirees shoppers.
\item
  We found that the high spending on chips for mainstream young
  singles/couples and retirees is due to more of them than other buyers.
\item
  Mainstream, mid-age, and young singles and couples are also more
  likely to pay more per packet of chips.
\end{enumerate}

It is indicative of impulse buying behavior.

\begin{enumerate}
\def\labelenumi{\arabic{enumi}.}
\setcounter{enumi}{3}
\item
  We've also found that Mainstream young singles and couples are 23\%
  more likely to purchase Tyrrells chips than the rest of the
  population.
\item
  The Category Manager may want to increase the category's performance
  by off-locating some Tyrrells and smaller packs of chips in
  discretionary space near segments where young singles and couples
  frequent more often to increase visibility and impulse behavior
\end{enumerate}

\end{document}
